%%%%%%%%%%%%%%%%%%%%%%%%%%%%%%%%%%%%%%%%%
% 
% K-CV -- Klenske Curriculum Vitae
% 
% Simplified Friggeri CV 
% by Edgar Klenske (ed.klenske@gmx.de)
% https://github.com/eklenske/CV
%
% Forked from:
% Jelmer Tiente 
% https://github.com/JelmerT/CV
%
% Based on original work by:
% Adrien Friggeri (adrien@friggeri.net)
% https://github.com/afriggeri/CV
%
% License:
% CC BY-NC-SA 3.0 (http://creativecommons.org/licenses/by-nc-sa/3.0/)
%
%%%%%%%%%%%%%%%%%%%%%%%%%%%%%%%%%%%%%%%%%

\documentclass[]{k-cv} % Add 'print' as an option into the square
                       % bracket to remove colors from this template
                       % for printing
\usepackage{ctex}


\begin{document}
\header{Oscar}{Andell}{} % Your name and current job

%-------------------------------------------------------------------------------
%	SIDEBAR SECTION
%-------------------------------------------------------------------------------

\begin{aside} % In the aside, each new line forces a line break
\includegraphics[width=4cm, height=4cm]{oscar2.jpg}
\begin{flushleft}
\section{About Me}
\color{gray}I am an aspiring engineer and software developer. I have a great interest in programming combined with a passion for history, fiction, internationalism and traveling. 
\end{flushleft}
\section{Date of Birth}
\color{gray}1994-07-29
\section{Contact}
Nanna Svartz gata 11
583 28 Linköping
Sweden
~
Oscar@Andell.eu
\href{http://Andell.eu/Oscar}{Andell.eu}
+46 765614624

\section{Programming}
\color{black}\textit{Java}
\color{gray}Proficient, Android development, OOP
~
\color{black}\textit{JavaScript}
\color{gray}Proficient, Node.js, React, CSS, HTML, Web development
~
\color{black}\textit{Python}
\color{gray}Basic proficiency, Flask
\end{aside}

%-------------------------------------------------------------------------------
%	EDUCATION SECTION
%----------------------------------------------------------------------------------------

\section{Education}

\begin{entrylist}
%------------------------------------------------
\entry
{Aug 2015 \newline\to now}
{Master of science in IT}
{Linköping University, Sweden}
{Master program \emph{International Software Engineering} which focuses on IT and software development in an international setting.}

%------------------------------------------------
\entry
%sizebox
{Feb 2019\newline\to Jul 2019}
{Exchange student, China}
{Harbin Institute of Technology 哈工大, China}
{For my master in International Software Engineering I spent one semester in the
Chinese city of Harbin (哈尔滨) studying software engineering with an
international class with Chinese and foreign students.}
%------------------------------------------------
\end{entrylist}

\section{Work Experience}
\begin{entrylist}

%------------------------------------------------
\entry
{Sep 2018\newline \to Dec 2018}
{MindRoad AB}
{Linköping}
{\emph{Internship}\\ At the consulting company Mindroad I worked with serval different tasks,
including unit-testing, proofing and testing web development courses, as well as
developing and demoing a prototype for a potential future project.}
%------------------------------------------------
\entry
{Jun 2018 \newline\to Aug 2018}
{Saab Aeronautics}
{Linköping}
{\emph{Summer Internship} \\During my internship I worked as a software developer in the Gripen project. Together with another intern I developed tools for storing, visualizing and analysis of software testing data.}
%------------------------------------------------
\entry
{Aug 2017\newline\to Dec 2018}
{Linköping University}
{Linköping}
{\emph{Tutor in problem-based learning (PBL)} \\
Beside my own studies I tutored a group of first year IT students in PBL.}

\entry
{Jun 2017\newline\to Aug 2018}
{Saab Training \& Simulation}
{Huskvarna}
{\emph{Summer Internship} \\
In the summer of 2017 i worked at Saab Training \& Simulation in Huskvarna.
During my employment i worked with automating the process of migrating older
data and schematics to a newer system, as well as a testing assistant during
testing of new simulations.}

\entry
{Nov 2013\newline\to Aug 2015}
{Swedish Armed Forces}
{Skövde}
{\emph{Soldier at the logistic regiment in
Skövde.}}


\end{entrylist}
\clearpage
\smallheader{Oscar}{Andell}

\begin{aside}
~
~\color{gray}
\section{Other Technologies}
Git, Docker, MongoDB, MySQL, NGINX, Latex, REST, Desgin Patterns, Architecture, Testing

\section{Languages}
\textit{Swedish}, native language
\textit{English}, fluent
\textit{Spanish}, conversational
\textit{Chinese}, basic knowledge
\section{Miscellaneous}
Driver's license with the B and BE qualification.
\end{aside}
\section{Projects and Extra curricular activities}
\begin{entrylist}
%------------------------------------------------
\entry
{\to now}
{\href{https://play.google.com/store/apps/details?id=se.oandell.riksdagen&hl=en}{Riksdagskollen}}
{Riksdagen}
{Together with my friend and fellow student I created the Android application \textit{Riksdagskollen}. This application provides an accessible way to follow our elected representatives and view the work that goes on in the Swedish parliament. The app is available on \href{https://play.google.com/store/apps/details?id=se.oandell.riksdagen&hl=en}{Google Play} and has received a lot of positive feedback from users.}
%------------------------------------------------
\entry
{Sep 2019\newline\to Dec 2019 }
{Software Engineering Project Course}
{Linköping University}
{\emph{Company role: Software Architect}\\ With a mock company of 25 students we developed an open source product for the company \textit{Ericsson}. My main contributions were designing the architecture, act as a stand-in lead developer and setting up automated tests and continuous integration.}
%------------------------------------------------
\entry
{\to now}
{\href{http://Andell.eu/Oscar}{Andell.eu}}
{Personal website}
{This website is the product of me experimenting and trying to learn more about
Javascript, Rest API:s, server management, hosting and web development in general. React frontend with a flask backend serving data from a MySQL database. Contains user, login and text edit functionality.}

\entry
{Aug 2017\newline\to Jun 2018}
{Web committee for Datateknologsektionen}
{Linköping University}
{Member of the student associations web team. React development of a web application for booking/overseeing the sections storage on
campus.}
\end{entrylist}





\clearpage



%-------------------------------------------------------------------------------
%	PUBLICATIONS SECTION
%-------------------------------------------------------------------------------

%\section{publications}
%\bibentry{Near a Raven -- A Critical Review of Poe's Works}{John 
%Smith}{Sprinkler Publishing, New York, 2010}

\end{document}
